\documentclass[xetex,aspectratio=169]{beamer}

\usepackage{res/lections}

\preamble

\title[История C]{Историческая справка:\\особенности языка C и его экосистемы}

\begin{document}

\titleslide

\tocslide

\begin{frame}{Dennis Ritchie: The Development of the C Language}
	\url{https://www.bell-labs.com/usr/dmr/www/chist.html}
	\begin{itemize}
		\item Сеттинг
		\item Корни в других языках
		\item Использование сейчас
	\end{itemize}
	Мы периодически будем обращаться к этому тексту
\end{frame}

\section{Мотивация}

\begin{frame}{Поколения ЭВМ}
	\begin{enumerate}
		\item 1940е--1950е. Электромагнитные реле и радиолампы: $10^5$ вт, большие машинные залы; доступны военным и работающим на них физикам
		\item 1950е--1960е. Полупроводники (транзисторы, диоды): $10^4$ вт, несколько стоек; доступны крупным учреждениям, банкам
		\item 1960е--1970е. Интегральные схемы: $10^2-10^3$ вт, одна или несколько стоек; доступны небольшим учреждениям и лабораториям
		\item 1970е--1980е--н.в. Микропроцессоры в одной интегральной схеме: $10-10^2$ вт, небольшой корпус, доступны мелким организациям, позже --- физическим лицам
	\end{enumerate}
\end{frame}

\begin{frame}{Сеттинг}
	\begin{block}{Подход 1960-х вцелом}
		\begin{itemize}
			\item Мэйнфреймы типа IBM/360 или GE-645
			\item Языки программирования типа PL/I
			\item ОС типа OS/360 или Multics
			\item Пакетное управление заданиями в духе JCL
			\item Нет универсальных интерактивных оболочек
		\end{itemize}
		Всё тяжёлое, толстое и сложное
	\end{block}
\pause
	\begin{block}{Изменение подхода 1970-х}
	\begin{itemize}
		\item Более простые \emph{и дешёвые} миникомпьютеры типа DEC PDP-7
		\item Более вольное их использование
		\item Зоопарк архитектур и семейств
	\end{itemize}
\end{block}
\end{frame}

\begin{frame}{Появление Unix}
Киллер-фичи Unix:

\begin{itemize}
	\item Древовидная иерархическая файловая система
	\item Агностический подход к данным в файлах: файлы стали просто последовательностями байтов, а раньше программисты работали с форматированными датасетами, это было часто быстрее, но сложнее
	\item Интерактивная пользовательская командная оболочка
	\pause
	\item Ну и ещё там всякое, о чём позже\ldots
\end{itemize}

\end{frame}

\begin{frame}{До и после Unix}

\begin{itemize}
	\item Multics умела и предоставляла многие упомянутые возможности, но была слишком сложной, а пользователи мечтали о минимализме
	\item Подходы Unix актуален и по прошествии 50 лет: его аспекты позже проявлялись в дизайне, например DOS и Windows
\end{itemize}

Милый документальный фильм от AT\&T:
\begin{itemize}
	\item \url{https://youtu.be/tc4ROCJYbm0}
\end{itemize}

\end{frame}

\section{Ранняя история}

\begin{frame}{Популярные языки 1960-х и ранее}
	\begin{itemize}
		\item Fortran: один из первых, высокоуровневый, быстрый, вычислительный
		\item COBOL: для деловых задач
		\item PL/I: общего назначения, мощный и сложный, подходил для системного программирования
		\item Языки ассемблера для разных архитектур, не переносимые
	\end{itemize}

\pause

	\begin{itemize}
		\item \emph{Все перечисленные --- не только языки ассемблера ---} не очень-то переносимый
		\item Неструктурированные языки, что ухудшало качество кода --- «спагетти-код» тяжело сопровождать
	\end{itemize}
\end{frame}

\begin{frame}{Появление C}
	\begin{itemize}
		\item Компилируемый
		\item Структурный
		\item Хорош в системном программировании
		\item Достаточно простой
        \item Переносимый
	\end{itemize}
\end{frame}

\begin{frame}{Что такое структурное программирование?}
	Программа состоит из:
	\begin{itemize}
		\item Блоков с последовательностью операторов
		\item Циклов
		\item Ветвления (\mintinline{C}|if| --- \mintinline{C}|else| --- \ldots)
		\item Перечисленное можно комбинировать
	\end{itemize}
	\pause

	\emph{Теорема B\"{o}hm-Jecopini}: перечисленное достаточно для того, чтобы выразить любой алгоритм в смысле полноты по Тюрингу
	\pause

	Вдобавок:
	\begin{itemize}
		\item \mintinline{C}|goto| есть, но не приветствуется
		\item Процедуры (функции)!
		\item Лексическая область видимости переменных (а не как в Basic или Python!)
	\end{itemize}
\end{frame}

\begin{frame}{Что ещё глянуть?}
	\begin{itemize}
		\item Notes on Structured Programming. By Prof. Dr. Edsger W. Dijkstra — T. H. Report 70-WSK-03 Second Edition April 1970
		\item Dijkstra: EWD 215: A Case against the GO TO Statement (PDF).
	\end{itemize}
\end{frame}

\begin{frame}{Предки C}
	\begin{itemize}
		\item 1960: \href{https://en.wikipedia.org/wiki/ALGOL_60\#Code_sample_comparisons}{Algol-60}
		\item 1963: \href{https://en.wikipedia.org/wiki/CPL_(programming_language)\#Example}{CPL}
		\item 1967: \href{https://en.wikipedia.org/wiki/BCPL\#Examples}{BCPL}
		\item 1969: \href{https://en.wikipedia.org/wiki/B_(programming_language)\#Examples}{B}
		\item 1972: \href{https://en.wikipedia.org/wiki/C_(programming_language)\#\%22Hello,_world\%22_example}{C}
	\end{itemize}
\end{frame}

\section{C и Unix}

\begin{frame}{C and Unix evolved together}
	\begin{itemize}
		\item C был переносим (не пользовался специфической функциональностью разных архитектур)
		\item C был достаточно прост, чтобы быстро реализовывать порождение кода для новых архитектур
		\item C хорошо подходил для системного программирования
	\end{itemize}

	\pause
    Изначально Unix написан на языке ассемблера, но в начале 1970-х его в значительной степени переписали на C, и это до сих пор одна из причин его популярности! Сейчас Unix-подобные системы на серверах, сетевом оборудовании, ПК, мобильных устройствах и т.д.
\end{frame}

\section{Актуальная история}

\begin{frame}{1980s--1990s}
	\begin{itemize}
	\item Дешёвые ПК
	\item Интернет
	\end{itemize}
\end{frame}

\begin{frame}{1990s--2000s--н.в.}
	\begin{itemize}
		\item Много мобильных и встроенных архитектур
		\item Параллельные архитектуры
	\end{itemize}
\end{frame}

\postamble

\end{document}
